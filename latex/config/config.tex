\usepackage{fancyhdr} %for headers
\usepackage[margin=1.0in]{geometry} % margin size (can change units)
\usepackage{setspace} % setting spaces in title
\usepackage{amssymb}

\usepackage{cite} % for references

\usepackage{graphicx}

\usepackage{floatrow} % to add side by side tables with figures

\usepackage{booktabs} % adding in table support

\usepackage{siunitx} % adding nice scientific notation
\usepackage{xfrac} % adding in slanted fractions

\renewcommand{\arraystretch}{1.2} % make tables a bit larger (vertically)
\setlength{\tabcolsep}{5 pt}	  % make the table width larger
\newdimen\heavyrulewidth
\newdimen\lightrulewidth
\heavyrulewidth=.12em  %make the top rule of a table heavy
\lightrulewidth=.05em  %and the line below the titles a bit lighter

\pagenumbering{arabic}

% Table float box with bottom caption, box width adjusted to content
\newfloatcommand{capbtabbox}{table}[][\FBwidth]

\usepackage[pdftex]{hyperref} %to add hyperlinks to the referenced pages for easier navigation

\hypersetup{
	colorlinks,
	allcolors=black
}

\headheight = 18pt

\widowpenalties 1 10000
\raggedbottom


%% Packages to include
\usepackage{amsmath}

%% creating step by step environment
\usepackage{enumitem}
\newlist{steps}{enumerate}{1}
\setlist[steps, 1]{label = Step \arabic*}

\usepackage{caption}
\usepackage{subcaption}
\usepackage{mathtools}

% numbering referenced labels in align environments
\newcommand\numberthis{\addtocounter{equation}{1}\tag{\theequation}}

%Adding in nice abs symbols
\usepackage{mathtools}
\DeclarePairedDelimiter\abs{\lvert}{\rvert}
\DeclarePairedDelimiter\norm{\lVert}{\rVert}

\newcommand{\e}[1]{\times10^{#1}}
\usepackage[labelfont=bf]{caption}

