Nuclear magnetic resonance (NMR) arises from an intrinsic property of nuclei called spin. This property can be used to determine the molecular structure and purity of various samples. In this report, NMR is used to investigate the properties of various samples and to determine the the spin-spin relaxation times ($T_2$ and $T_{2}^*$) and spin-lattice relaxation ($T_1$) times. The samples explored are copper-sulphate doped water as a calibration sample, ethanol, and rubber. Free induction decay, inversion recovery, saturation recovery  and Hahn spin echo techniques are used to determine the time constants. Ethanol is found to have time constants of $T_1 = 2357\pm 82$ ms and $T_2=42.41\pm 3.38$ ms, and rubber has $T_1 = 38.48\pm 2.09$ ms and $T_2 = 4.80\pm1.11$ ms. These results demonstrate that for our samples, $T_1$ is approximately an order of magnitude larger than $T_2$, and the time constants for ethanol are an order of magnitude larger than that of rubber. 