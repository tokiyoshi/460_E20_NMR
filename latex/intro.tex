Nuclear magnetic resonance (NMR) is the phenomena describing how the nuclei in a static magnetic field interacts with external effects, such as an applied magnetic field. This interaction is shown in the absorption and emission of radiation at characteristic rates. This phenomena is fundamental and is relatively simple -- to the degree that it can be derived from the fundamental principles of quantum mechanics. Since this phenomena is fundamental and universal, it appears in several contexts and is of use to many different fields. As a result, the term NMR can mean different things to different people. One such use is MRI, originally named NMR imaging, which is arguably the most ubiquitous and widespread usage of ``quantum technology''. However, the context which we will be investigating it a more common lab experiment known as NMR pulse spectroscopy. \\

NMR generally involves applying a strong magnetic field to align nuclear magnetic moments under study, then a perpendicular magnetic field is applied to perturb this overall magnetic alignment. The study of the recovery to alignment can reveal details of the material under study. In pulse NMR spectroscopy this perpendicular magnetic field is applied for a specific amount of time to ``push'' the overall magnetic moment of the material into a known orientation, and then the perpendicular field is turned off and the recovery process of the material is observed to glean details of the material under study.\\

In this experiment three different pulse schemes are applied to three different materials to study their different recoveries and from this we can investigate the different materials and the different schemes used to make them. These materials are a calibration sample (doped water), ethanol, and rubber. The experiment also includes an investigation of the frozen doped water.\\

\newpage



%Nuclear magnetic resonance spectroscopy is an imaging technique which relies interfacing with an intrinsic nuclear property known as spin by applying a magnetic field. These magnetic fields can be oscillated in particular ways to create resonance conditions within the material. These different resonances are based off of fundamental quantities so experimentally finding this condition allows for people to learn about the material

%Magnetic nuclear resonance occurs because the nuclei of many atoms have a property called spin. Nuclei with spin also have an angular momenta, which is equivalent to visualizing a classical model consisting of a small permanent magnetic rotating about its long axis. The axis in which it is rotating has a magnetic moment pointing along that axis. When a constant magnetic field is applied, this angular momentum, which is proportional to the moment, causes the nuclei to precess about the magnetic field. The frequency of this precession is called the Larmor frequency and it is proportional to the strength of the magnetic field and the gyromagnetic ratio. The gyromagnetic ratio is unique for each atom. This property is often used in NMR spectroscopy, where scientists seek to determine the molecular structure of compounds and/or the purity of a sample (University of Sydney). In addition to research, NMR is also used in everyday life. For example, it is used for magnetic resonance imaging scans in the medical industry and for quality control in the food industry (University of Sydney). \\

%In our experiment, we work with three different samples: rubber, ethanol and water. For each sample we apply a constant magnetic field, $B_0$. 

%As mentioned above, this causes the nuclei of the atoms to start to precess about the magnetic field. Due to quantum mechanics, the angle between the magnetic moment and the magnetic field vector can only occur at specific quantized values. For protons, there are only two angles, which are the states called spin up and spin down. Spin up occurs when the protons are aligned parallel with the field and spin down is when they are anti-parallel to the field. The nuclei in spin up exist in a lower energy state than the spin down nuclei because it takes more energy for the spins not to align. 

%Due to thermal physics, there is a distribution of particles in either spin up and spin down. However, a large number of the nuclei will end up in the spin up state due to it being the lower energy state. This will create a macroscopic magnetization vector along the \textit{B} field. To find out information about the sample, another magnetic field is applied, $B_1$. The applied field rotates about the $B_0$ axis at the Larmor frequency so that constant torque is applied to the magnetization vector. The result is that the magnetization rotates about $B_1$. In our experiment, the $B_0$ field is applied along the z-axis and $B_1$ is applied so that the magnetization vector rotates towards the xy plane. Applying the $B_1$ field for different amounts of time corresponds to the final direction of the magnetization vector. A 90 degree pulse occurs when $B_1$ is applied until the magnetization vector is directed along the x-axis. Our detector is set up to measure NMR signals along the x-axis. The detector measures the emf due to Faraday's law, which is caused by the spins precessing about the \textit{B} field. The recorded signal is called the free induction decay (FID). The time for the signal to decay is called the spin-spin relaxation time, $T_2$. The signal decays due to the nuclei spinning at different frequencies. This can be caused by the thermal distribution of the sample and/or differences in the magnetic field, $B_0$. After the magnetization vector has been directed into the xy plane and the applied field is turned off, the spins start to dephase, which causes the magnetization vector along the x axis to decrease. This results in a decrease of our detected signal. The differences in the magnetic field could be caused by intermolecular forces or by imperfections in $B_0$. $T_{2}^*$ takes into account both of these processes whereas $T_2$ only takes into account the intermolecular forces. \\

%There is another phenomenon occurring after the $B_1$ field has been turned off. The energy that the nuclei gain from the $B_1$ pulse is returned to the system, or lattice. As this happens, the spins start to return to their equilibrium populations of spin up and spin down and the magnetization vector starts to point along $B_0$. The time for this regrowth is called $T_1$, the spin-lattice relaxation time. The time varies based on the intermolecular forces because the energy can be given up in several ways, such as molecular rotation, diffusion or lattice vibration (ref). Therefore, we'd expect $T_1$ to be shorter for liquids due to the additional ways that the nuclei can release their energy. \\

%The purpose of this lab is to explore different methods to determine $T_{2}^*$, $T_2$ and $T_1$. To determine $T_{2}^*$, a FID was taken for each sample. To determine $T_2$, the Hahn Echo technique is used, which involves applying a 90 degree pulse and then a 180 degree pulse to allow for the signal to correct for the $B_0$ inhomogeneities. To determine $T_1$, a 90 degree pulse is applied and then some time later another 90 degree pulse is applied to see how much of the magnetization vector has regrown along the z-axis. The aim is to see how the relaxation times vary based on the sample and to use this information to think about the intermolecular forces for each.

% \begin{itemize}
% \item "NMR is possible because nuclei of many atoms possess magnetic moments and angular momenta" - for fun: electrons possess moments about 1000 times larger than the largest nuclear moment
% \item A magnetic moment interacts with a static magnetic field in such a way that the field tries to force the moment to line up along it
% \item The significance of the angular momentum, which is proportional to the moment, is that it makes the nuclei precess about the spin axis rather than oscillating in a plane like a compass needle like a spinning top precessing about earths magnetic field
% \item the precession frequency of the moment is proportional to the uniquely determined gyromagnetic ratio and the strength of the magnetic field. It is given by the larmor relation w0 = gamma B0
% \item gamma is the proportionality constant between the moment and angular momentum
% \item In a given magnetic field, the precession frequency is different for every distinct nucleus because each has a uniquely defined gamma
% \item QM requires the that the orientation of a magnetic moment with respect to the field to be quantized. 
% \item positive moment will always want to align parallel with the field - lower energy state
% \item the lower energy state nucleus can transition to a higher energy state by receiving the right amount of energy - hbar w, w is the angular velocity of the applied rf field. If the frequency of the radiation applied is too low or too high then the nucleus will not undergo the transition. 
% \item **applied rf radiation provides a rotating magnetic field component and when it is rotating at the same angular velocity, it applies torque to the nucleus causing its angle with the static field to change
% \item all nuclei would align with the field and be in the lower energy state if it were not for thermal motion of molecules and atoms counteracting the effect of the field. The population ratio of the two states depends on the samples absolute temperature and the strength of the magnetic field B by the Boltzman factor: exp(-mu times B /(kT). For NMR experiments at room temperature, this factor differs from unity yb only $10^-5$ or -6 so that most of the protons are randomly distributed and their effects cancel. 
% \item Large number of protons do line up and provide a macro magnetization whose precession can be detected under certain conditions. 

% \item if somehow the magnetization is rotated away from the field, it can relax back to thermal equilibrium, towards the field, only by giving up quanta of energy to the surroundings, such as the KE of a molecule in multiples of hf = hbarw. The surroundings can absorb this energy from the nuclei are called the lattice. 

% \item problem with NMR is small signal to noise ratio. the signal to noise ration depends on the design of the electronics as well as the kind and number of nuclear spins. Sources of noise include: fluorescent lights, electronic equipment such as the minicomputer which is used to improve the signal to noise ratio. Averaging multiple experiment runs. S/N increases as root t, time required to take data

% \item first nmr experiments were the continuous wave exps by bloch, purcell and others. first pulse experiments came shortly after, hahn published spin echo paper in 1950. for a decade and a half, pulse was used to determine t1 and t2.Lowe and Norberg NMR frequency spec is related to an fid by fourier. line narrowing exp with solid sample have become possible.
% \item the magnetization dephases in the xy plane because of field inhomogeneity - some nuclei will precess faster than others and because thermal equilibrium is being reestablished. If one waits long enough, the magnetization will reestablish itself along the z direction in equilibrium with the applied field with a time constant called the spin lattice or longitudinal relaxation time t1
% \item the decay rate for the magnetization in the xy plane is usually larger than or equal to the decay rate for the recovery of the magnetization along the z direction
% \item because the pickup coil is sensitive to only the component of magnetization in the xy plane, the magnetization being reestablihsed along the z' axis due to the t1 process is not detectable until it is rotated away from the z' axis. Therefore, a second pulse is always necessary to mesaure the spin lattice relaxation time t1. The basic idea is to perturb the magnetization from its equilibrium state with one pule and examine its recovery along the z' axis after a variable delay with another pulse, such as a 90 degree pulse 

% \item  90 degree pulse in an inhomogenous magnetic field: the total magnetization vector is the sum of the smaller magnetization vectors arrising from a small volume experiencing a homogeneous field. After a 90 pulse, each of these components of the magnetization will precess with its own characteristic larmour frequency. Therefore, the magnetization from those portions of the sample with slighlty larger fields will precess quicker than those which are in the smaller fields. As a result, the different contributions of the magnetization will get out of phase with each other. The contribution to the magnetization which arises from one small segment of the sample experiencing a homogeneous field is called a spin isochromat. Implication is that all nuclear moments in that segment will precess with the same frequency. the pickup coil in the xy plane will decay as the spin isochromats fan out - free induction decay. 

% \item the time constant which describes the decay of magnetization in the xy plane is t2star. We define an intrinsic relaxation time which is characteristic of the magnetization decay in one of the spin isochromats without any field inhomogeneity effects and this is the spin-spin or the transverse relaxation time. t2star will increase until it becomes equal to t2 if the field homogeneity is improved. When t2star is dominated by the magnetic field inhomogeneity we get little info about the sample. 

% \item while this is going on, spin lattice relaxation is also taking place. the spin spin relaxation does not involve any exchange of energy with the world outside the spin system whereas the spin lattice relaxation depends on an outside agent accepting the energy from the spin system so that the latter can relax towards the thermal equilibrium state given by the Boltzman populations. lattice can be of many forms usch as molecular rotation, diffusion, or lattice vibration. spin relaxation can be very weak in the absense of molecular motion or paramagnetic ions as in many rigid solids but can be very strong in liquids and in some solids which exhibit molecular motion. The spin-spin relaxation process contains a contribution from the spin lattice process so that t1>>t2star for solids and t1 is close to t2star for most liquids. 

% \item spin-echo techniques: used to remove the effect of the applied field inhomogenity. Apply a 90 degree pulse - shortly after, the spin isochromats will have dephased in the xy plane. As a result, there is no net M, although the individual spin isochromats have not dephased. 180 degree pulse is applied. anything in the z will be inverted to -z and doesn't matter. spin isochromats which have gotten ahead of the average spin isochromats by a certain angle and nnow behind the average pack by the same amount and vice versa. Therefore, following a 180 pulse, the spin isochromats begin to rephase. - get spin echo by applying 180 to y' axis in the rotating frame so that the refoccussing will take place along y'axis so that the echo will have the same sign as the fid. 

% \item one of the problems in accumulating fids is that enough time must elapse between the fids in order for the magnetization to build up along the z-axis so that the signal is relatively large. if t1 is approx t2star like in most liquids then this isn;t a problem but if t1 is larger than t2star (solids) then time intervals on the order of t1 are still needed between the fids even if the fid has disappeared long before. 
% \end{itemize}