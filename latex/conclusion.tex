In this experiment we explored the Nuclear Magnetic Resonance for three samples; a calibration sample of doped water, representative liquid sample of ethanol, and a representative solid sample of rubber. Using methods including free-induction decay, saturation-recovery, inversion-recovery and Hahn echo, the time constants $T_1$, $T_2^*$ and $T_2$ are calculated for all three samples.\\

The ethanol sample has time constants $T_1 = 2357\pm 82$ ms and $T_2=42.41\pm 3.38$ ms, while rubber sample has constants $T_1 = 38.48\pm 2.09$ ms and $T_2 = 4.80\pm1.11$ ms. \\

Using inversion-recovery sequence a more accurate method for calculating $T_1$ than the zero-crossing method or saturation-recovery method. \\

From the presented results, we can conclude that $T_1$ is approximately an order of magnitude larger than $T_2$ and the time constants for ethanol are an order of magnitude larger than that of a rubber due to the differences in inter-molecular forces.\\
